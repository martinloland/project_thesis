\chapter{Theory}
This chapter will develop the relationships between rigid motion, kinematics, velocity and dynamics. A strong theoretical foundation of these ideas are elementary for a successful modelling of dynamic behaviour. A successful framework will rely heavily on the theoretical background so the content of the coming chapters will strongly build on the theory presented in this part.

\section{Rigid motion}
\subsection{Position, rotation and translation}
\subsubsection{Rotations in three dimensions}

If a reference frame $o_0 x_o y_o z_o$ (denoted $O_0$) is rotated in all three dimensions to $O_1$ the \textit{rotation matrix} can be described as the projection of the unit axis in $O_1$ onto $O_0$ as in \eqref{rotmatrix}. This matrix will be used to express point $p^1$ in the reference frame of $O_0$, \eqref{pointrotation}. The same rotation matrix can also be used as an operator to rotate a vector in a fixed reference frame.
\begin{equation}\label{rotmatrix}
R^0_1=\begin{bmatrix}
x_1\cdot x_0 & y_1\cdot x_0 & z_1\cdot x_0\\ 
x_1\cdot y_0 & y_1\cdot y_0 & z_1\cdot y_0\\ 
x_1\cdot z_0 & y_1\cdot z_0 & z_1\cdot z_0
\end{bmatrix}
\end{equation}

\begin{equation}\label{pointrotation}
p^0 = R^0_1p^1
\end{equation}

If $T^0$ is a linear transformation in $O_0$, the same transformation can be expressed in terms if another reference frame $O_1$ as $T^1$ by the help of \eqref{similaritytrans}. This is described as a \textit{similarity transformation} and can be used to express transformation in different reference frames.

\begin{equation}\label{similaritytrans}
T^1 = (R^0_1)^{-1}T^0 R^0_1
\end{equation}

\subsubsection{Composition of rotational transformations}

If $R^i_j$ and $R^k_l$ represents rotational transformations between reference frames \textit{i, j} and \textit{k, l}. Then point $p^l$ can be described in terms of $O_i$ as in \eqref{rotationcomposition}. This will be rotation with respect to the \textit{current frame}, which will change as as the transformation happen. If a \textit{fixed axis rotation} is wanted instead, the rotation composition has to be pre multiplied instead of post multiplied, shown in \eqref{rotationcomposition_pre}.


\begin{equation}\label{rotationcomposition}
p^i = R^i_j R^k_l p^l
\end{equation}

\begin{equation}\label{rotationcomposition_pre}
p^i = R^k_l R^i_j p^l = R R^i_j p^l
\end{equation}

\subsubsection{Parametrisation of rotations}

In equation \eqref{rotmatrix} the transformation were described in terms of nine independent variables. A rotation in 3D can at most have three independent variables and it is therefore convenient to perform a parametrization to get three independent variables. This can be done by using \textit{euler angles}, \textit{roll pitch yaw} or \textit{axis/angle} representation. The euler angle rotation is rotation about ZYZ axis by an amount $\phi, \theta, \psi$ respectively, see \eqref{eulertrans}. \textit{C} and \textit{s} represents the cosine and sine functions respectively.

\begin{align}\label{eulertrans}
R_{ZYZ} &= R_{z,\phi}R_{y,\theta}R_{z,\psi} \\
&= \begin{bmatrix}
c_\phi c_\theta c_\psi - s_\phi s_\psi & -c_\phi c_\theta s_\psi - s_\phi c_\psi & c_\phi s_\theta\\ 
s_\phi c_\theta c_\psi + c_\phi s_\psi & -s_\phi c_\theta s_\psi + c_\phi c_\psi  & s_\phi s_\theta \\ 
-s_\theta c_\psi & s_\theta s_\psi & c_\theta
\end{bmatrix}
\end{align}

\subsection{Rigid motion}
\subsection{Spaces}

\section{Kinematics}
\lipsum[1]
\subsection{Denavit-Hartenberg convention}
\subsection{Forward and inverse kinematics}

\section{Velocity kinematics}
\lipsum[1]
\subsection{Jacobian}
\subsection{Force and torque relationships}
\subsection{Inverse velocity and acceleration}

\section{Dynamics}
\lipsum[1]
\subsection{Euler-Lagrange equations}
\subsection{Kinetic and potential energy}
\subsection{Newton euler formulation}