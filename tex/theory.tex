\newgeometry{top=0.0cm, bottom=2.5cm}
\vspace{-0.9cm}
\chapter{Theory}\label{secTheory}

\vspace{-0.8cm}

The following chapter will develop the most used expressions for calculations of dynamic behaviour in a linked mechanism. Most of the notation and conventions will follow \textit{Robot Modeling and Control} by Spong (2006)\cite{spong}. 

A robotic manipulator consists of multiple links and joints, the position and orientation of these links can be represented as multiple \glsf{frame}\textit{s}. Sometimes \textit{reference frame} will be used to describe the same thing. The frame is defined with three orthogonal unit vectors and position relative to another frame.

The first frame in a serial manipulator is called the \glsf{inertialframe} and denoted $O_0$ (big letter O and subscript zero). It is defined to have position and rotation equal to zero, see \figref{frames}. The next frame in the chain is $O_1$ and is connected to $O_0$ through link 1.

Joint 1 ($q_1$ in the figure) is situated in the origo of $O_0$, while joint 2 is located at $O_1$. In general, link $n$ connects to joint $n$ and $n+1$. Joint $n$ is in the same position as $O_{n-1}$ so that when joint $n$ is actuated, link $n$ and frame $n$ is moved accordingly. The variable $q$ will be used to denote the joint variable. This is the angular displacement for a revolute joint or linear displacement for a prismatic (linear) joint.

\begin{figure}[h!]    
    \centering           
    \def\svgwidth{.7\columnwidth}
    \input{inkscape/frames.pdf_tex}
    \caption{Frames, links and joints for a revolute serial manipulator}
    \label{frames}
\end{figure}
\restoregeometry


\section{Rigid motion}
\subsection{Position and rotation}

\subsubsection{Rotation in three dimensions}

Frame $O_1$ in \figref{frames} has a rotation with respect to $O_0$, in this case a positive z-rotation $q_1$, using the right hand rule. It is helpful to express an arbitrary rotation in all three dimensions and this is the intention of the rotation matrix, \eqref{rotmatrix}. This is a $3\times3$ matrix and can be found by taking the vector products of the unit vectors as in the equation. $R_1^0$ will describe the \textit{rotation matrix} from $O_0$ to $O_1$. By post multiplying $R_1^0$ with a position vector with respect to $O_1$, $p^1$, it is possible to obtain the coordinates to the same point $p^0$, but now with respect to $O_0$, see \eqref{pointrotation}.

This is only considering a rotation of the frame, not translation as will be described shortly. For a 2D scenario where only rotation along the z-axis is possible the matrix would reduce to \eqref{rot2d}

%If reference frame $O_0$ is rotated in all three dimensions to $O_1$ the \textit{rotation matrix} can be described as the projection of the unit axis in $O_1$ onto $O_0$ as in \eqref{rotmatrix}. This matrix will be used to express point $p^1$ in the reference frame of $O_0$, \eqref{pointrotation}. The same rotation matrix can also be used as an operator to rotate a vector in a fixed reference frame.
\begin{equation}\label{rotmatrix}
R^0_1=\begin{bmatrix}
x_1\cdot x_0 & y_1\cdot x_0 & z_1\cdot x_0\\ 
x_1\cdot y_0 & y_1\cdot y_0 & z_1\cdot y_0\\ 
x_1\cdot z_0 & y_1\cdot z_0 & z_1\cdot z_0
\end{bmatrix}
\end{equation}

\begin{equation}\label{pointrotation}
p^0 = R^0_1p^1
\end{equation}

\begin{equation}\label{rot2d}
R^0_1(q_1)_z=\begin{bmatrix}
cos\left ( q_1 \right ) & -sin\left ( q_1 \right ) & 0\\ 
sin\left ( q_1 \right ) & cos\left ( q_1 \right ) & 0\\ 
0 & 0 & 1
\end{bmatrix}
\end{equation}


\subsubsection{Composition of rotational transformations}

If $R^0_1$ and $R^1_2$ represents rotational transformations between reference frames \textit{0, 1} and \textit{1, 2}. Then point $p^2$ can be described in terms of $O_0$ as in \eqref{rotationcomposition}. In general, the rotation matrix from $O_i$ to $O_n$ can be found by \eqref{Rcompositioninf}.

\begin{equation}\label{rotationcomposition}
p^0 = R^0_1 R^1_2 p^2
\end{equation}
\vspace{-0.4cm}
\begin{equation}\label{Rcompositioninf}
R^i_n = \prod_{i=i}^{n-1} R^i_{i+1}
\end{equation}


\subsubsection{Parametrisation of rotations}

In equation \eqref{rotmatrix} the transformation were described in terms of nine independent variables. A rotation in 3D space can at most have three independent variables and it is therefore convenient to perform a parametrization to get three independent variables. Three methods are \textit{euler angles}, \textit{roll pitch yaw} or \textit{axis/angle} representation. This is left here as a reference for future studies, but since one link can only rotate around one axis the parametrisation is not needed.

\subsection{Rigid motion}

Rigid motion is a combination of rotation and translation. The matrix form is described as a \textit{homogeneous transformation} and is defined as a set of matrices as in \eqref{homeqn}. \textit{R} is the rotation matrix as described before and \textit{d} is the translation vector with respect to a chosen frame. The same features of composition as described in \eqref{Rcompositioninf} also applies to homogeneous transformations.

\begin{equation}\label{homeqn}
H = \begin{bmatrix}
R_{3x3} & d_{3x1}\\ 
0_{1x3} & 1_{1x1}
\end{bmatrix}=\begin{bmatrix}
r_{11} & r_{12} & r_{13} & d_{1}\\ 
r_{21} & r_{22} & r_{23} & d_{2}\\ 
r_{31} & r_{32} & r_{33} & d_{3}\\ 
0 & 0 & 0 & 1
\end{bmatrix}
\end{equation}


\section{Kinematics}

Kinematics describes the relationship between the joint angles and the position and orientation of the robot as a whole. The topic can be divided into \textit{direct} and \textit{inverse} kinematics. \textit{Forward} is in some places used instead of \textit{direct}. To define the direct kinematics, the Denavit-Hartenberg convention has been used.

\subsection{Denavit-Hartenberg convention}

The homogeneous transformation as described in \eqref{homeqn} is a matrix of six independent variables. By introducing two constraints on how the reference frames are defined with respect to each other the total number of independent variables can be reduced to four. This is the motivation behind the \textit{\gls{DH}} (D-H), first introduced in 1955 \cite{DH}. This convention also provides a common, well known framework for defining robotic systems among engineers and has been broadly adopted.

A D-H transformation is defined as a product of four basic transformations, \eqref{DH}. Where $a_i, \alpha_i, d_i $ and $\theta_i$ represents the link length, link twist, link offset and joint angle respectively, for link number \textit{i}. For more information on how these variables are assigned, the reader is encouraged to read some of the literature about it, \cite{Corke2007}.

The homogeneous transformation matrix from $O_0$ to the \gls{endeff} $n$ will then be the product sum of the transformations for each link, \eqref{DHsum}.

\begin{align}\label{DH}
\begin{split}
A_i &= Rot_{z,\theta_i}Trans_{z,d_i}Trans_{x,a_i}Rot_{x,\alpha_i} \\
&= \begin{bmatrix}
c_{\theta_i} & -s_{\theta_i}c_{\alpha_i} & s_{\theta_i}s_{\alpha_i} & a_{i}c_{\theta_i} \\ 
s_{\theta_i} & c_{\theta_i}c_{\alpha_i} & -c_{\theta_i}s_{\alpha_i} & a_{i}s_{\theta_i} \\ 
0 & s_{\alpha_i} & c_{\alpha_i} & d_i \\ 
0 & 0 & 0 & 1 
\end{bmatrix}
\end{split}
\end{align}

\begin{equation}\label{DHsum}
T^0_n=\prod_{i=1}^{n}A_i
\end{equation}

In essence, this establishes the forward kinematic relationship. Each link in the robot can be defined as in \eqref{DH} where either $\theta_i$ or $d_i$ is the varying variable depending on the joint is revolute or prismatic. Lastly, \eqref{DHsum} will give the position and rotation of the end effector with respect to $O_0$.

\subsection{Inverse kinematics}

Given a specific \gls{homo} $H$, the problem of inverse kinematics is to solve the forward kinematic equations for it's joint variables such that the transformation matrix of the robot arm equals that of $H$, \eqref{inverse_kin}.

\begin{equation}\label{inverse_kin}
T^0_n = H
\end{equation}

This approach will at most result in twelve nonlinear equations with twelve unknowns and can be very difficult to solve. There are multiple approaches for simplifying this problem, but since the topic is of limited interest in this thesis no more time will be spent on it.

\section{Velocity kinematics}

This section will describe how the velocity of any point on the robot can be defined with respect to the joint variables and it's velocities. Using what is called the \textit{Jacobian} is one method. It has range of applications when it comes to robotics, especially when it comes to inverse kinematic problems \cite{Duleba2013} and singularities \cite{Shi2012}. It is not directly needed for dynamics in direct kinematics, but a short description is included because of it's importance in the field.

\subsection{Jacobian}

The Jacobian establish the relationship between the velocity of a certain point on the robot $\xi$ and the joint velocities $\dot{q}$, \eqref{jacobian}. Where the linear and angular velocity of point $n$ with respect to the inertial frame $O_0$, $V^0_n$ and $\omega^0_n$ respectively is a product of the Jacobian $J$ and the joint velocities $\dot{q}$. 

\begin{align}\label{jacobian}
\begin{split}
\xi &= J\dot{q}\\
\begin{bmatrix}
V^0_n\\ 
\omega^0_n
\end{bmatrix}_{6\times 1}
&=
\begin{bmatrix}
J_v\\ 
J_\omega
\end{bmatrix}_{6\times n} \begin{bmatrix}
\dot{q}
\end{bmatrix}_{n\times 1}
\end{split}
\end{align}
\vspace{-0.2cm}
\begin{equation}\label{Jvel}
J_v = \begin{bmatrix}
J_{v_1} & J_{v_2} & ... & J_{v_n}
\end{bmatrix}
\end{equation}
\vspace{-0.2cm}
\begin{equation}\label{Jomega}
J_{\omega} = \begin{bmatrix}
z_0 & z_1 & ... & z_{n-1}
\end{bmatrix}
\end{equation}
\vspace{-0.2cm}
\begin{equation}\label{Jvi}
J_{v_i} = z_{i-1}\times (O_n - O_{i-1})
\end{equation}

Each part of the Jacobian, \Crefrange{Jvel}{Jvi} describes how the rotation of one joint affects the velocity at point $n$. The values can be found from the correct columns in the corresponding homogeneous transformation, \eqref{homeqn} describing that point. Specifically, $z=\begin{matrix}[
r_{13} & r_{23} & r_{33}]
\end{matrix}^T$ and $O=\begin{matrix}[
d_{1} & d_{2} & d_{3}]
\end{matrix}^T$.

\eqref{jacobian} provides the linear and angular velocity vectors of the point in question. The absolute velocity can then be calculated by \eqref{absvel}. Here the superscript $0$ are left out for simplicity and $r_n$ denotes the vector from the inertial frame to the point $n$.

\vspace{-0.5cm}
\begin{equation}\label{absvel}
v_{n}=V_n+\omega_n\times r_n
\end{equation}
\vspace{-0.5cm}

%\subsection{Force and torque relationship}
%
%The Jacobian in \eqref{jacobian} has a special property. If \\$F = \begin{matrix}[
%F_{x} & F_{y} & F_{z} & n_x & n_y & n_z]^T
%\end{matrix}$ is the force and torque vector at the end effector, the transpose of the Jacobian then describes the torques at each joint, \eqref{Jtorque}. This will prove helpful when the same equation can be used to find torques at the joints because of dynamic forces.
%
%\begin{equation}\label{Jtorque}
%\tau = J^T F
%\end{equation}

\subsection{Time difference in position vector}

An optional method to the Jacobian for calculating velocities and accelerations is the following. By looking at the difference in position over a finite time period, the velocity and acceleration can be calculated. The position vector would be $d$ from \eqref{homeqn}. Velocity and acceleration can then be calculated as in \eqref{d_der} and (\ref{d_derder}). $d_i$ and $v_i$ represent the position and velocity at time $t$ respectively. All values are with respect to the inertial reference frame.
\vspace{0.3cm}
\begin{equation}\label{d_der}
v_i = \dot{d} = \frac{d_{i}-d_{i-1}}{t_i-t_{i-1}}
\end{equation}
\vspace{-0.2cm}
\begin{equation}\label{d_derder}
a_i = \ddot{d} = \frac{v_{i}-v_{i-1}}{t_i-t_{i-1}}
\end{equation}
\vspace{-1cm}
\section{Dynamics}

The goal of this section is to develop at method to determine the force and torque in every joint as a result of the mass and mass distributions, orientation, velocity and acceleration of each link in the chain. To be able to do this, some physical properties has to be defined. The 2-link planar robot will be used as previous for visualization, but the concept apply to any other angular or prismatic configuration.

In \figref{speed} the angular velocity and acceleration along with the linear acceleration is presented. The linear velocity is left out as this does not have any effect on forces. The joint forces and torques together with the gravitational force is presented in \figref{forces}. If there are any end effector forces these will be denoted $f_e$ and $\tau_e$.
\begin{figure}[h!]    
    \centering           
    \def\svgwidth{.9\columnwidth}
    \input{inkscape/speed.pdf_tex}
    \caption{Velocity and acceleration properties}
    \label{speed}
    \vspace{-1cm}
\end{figure}

\begin{figure}[h!]    
    \centering           
    \def\svgwidth{.9\columnwidth}
    \input{inkscape/forces.pdf_tex}
    \caption{Forces and torques acting on a robot}
    \label{forces}
\end{figure}

\subsection{Mass and inertia matrix}

In \figref{forces} the center of gravity is visualized in the center of the link but this is not a requirement. The distance between the joint and center of gravity of one link ($r_{1,c1}$ and $r_{2,c2}$ in \figref{physical}) can be found by \eqref{masscen}. In practice this is determined by computer software for the specific link.

The inertia matrix is a $3\times 3$ matrix and can be found from \eqref{inertiacalc}. By calculating the inertia with respect to the center of gravity and utilizing symmetry in the geometry, the calculations can be simplified. This is left to the reader for investigation as it's outside the scope of this thesis.

\begin{equation}\label{masscen}
r_{i,ci} = \frac{1}{m_i}\int r \; dm
\end{equation}

\begin{equation}\label{inertiacalc}
I = \int r^{2}dm
\end{equation}

\begin{figure}[h!]    
    \centering           
    \def\svgwidth{.9\columnwidth}
    \input{inkscape/physical.pdf_tex}
    \caption{Physical dimensions and properties}
    \label{physical}
    \vspace{-1cm}
\end{figure}

%\subsection{Kinetic and potential energy}
%
%\todo[inline]{
%\textbf{Todo}
%
%- Maybye this section is not needed at all
%}
%
%Any applied force or torque to a system will result in a change in kinetic and potential energy. It is therefore valuable to describe these relations as this will aid in the process of finding the required torques at each link to achieve a specific movement.
%
%The kinetic energy $K$ of a single body is described in \eqref{kineticE}. Note that the linear and angular velocity, $v$ and $\omega$, are the velocities of the center of mass for that body with respect to the fixed frame. The frame attached to this point is called the \glsf{bodyframe} and are used for most of the calculations when it comes to dynamics. $\mathbf{I}$ represents the inertia tensor with respect to the \gls{inertialframe}. This tensor will change with the configuration of the body, but the inertia tensor with respect to the \gls{bodyframe} $I$ will stay constant. These are related as in \eqref{Irelated}.
%
%\begin{equation}\label{kineticE}
%K = \frac{1}{2}mv^Tv + \frac{1}{2}\omega ^{T}  I \omega
%\end{equation}
%
%\begin{equation}\label{Irelated}
%\mathbf{I} = RIR^T
%\end{equation}
%
%The potential energy is simply the mass times gravity and height, \eqref{potentialE}. $r_c$ is the vector from the reference frame to the mass center.
%
%\begin{equation}\label{potentialE}
%P = mg^Tr_c
%\end{equation}



\subsection{Newton-Euler formulation}

An interesting and maybe the most important question that can be asked when it comes to dynamics for a robot, is the following: What kind of torques on each link $\mathbf{\tau}(t)$ is needed to perform a particular movement $q(t)$? Or put in another way, given the current position, velocity and acceleration $q, \dot{q}$ and $\ddot{q}$, what is the resulting torque $\tau$ and force $f$ each joint have to endure? One method to solve this is with the \glsf{nef}. With reference to the generalized link $i$ in \figref{newtonEuler}, the  \gls{nef} will now be described in it's two steps, forward and backward recursion.

\begin{figure}[h!]    
    \centering           
    \def\svgwidth{\columnwidth}
    \begin{algorithm}[H]
\DontPrintSemicolon
	\textbf{Forward step}\;
	$\omega_0, \alpha_0, a_{c,0}, a_{e,0} = 0$  \tcp*[f]{joint 0 is always fixed}\;
	 \For{link i=1,2,...,n}{
		  solve $\omega_i$ from (\ref{omega})\;
		  solve $\alpha$ from (\ref{alpha})\;
		  solve $a_{e,i}$ from (\ref{a_end})\;
		  solve $a_{c,i}$ from (\ref{a_center})\;
	 }
	 \BlankLine
	\textbf{Backward step}\;
	$f_{n+1}, \tau_{n+1} = 0$  \tcp*[f]{if no load at end effector}\;
	$f_{n+1}, \tau_{n+1} = f_e, \tau_e$  \tcp*[f]{if load at end effector}\;
	 \For{link i=n,n-1,...,1}{
		  solve $f_i$ from (\ref{linear_equi})\;
		  solve $\tau_i$ from (\ref{angular_equi})\;
	 }
	 
 \caption{Newton-Euler method}\label{nefAlgo}
\end{algorithm}
    \caption{Newton-Euler variables for link i}
    \label{newtonEuler}
\end{figure}

On a general level, \gls{nef} works by setting the change in linear and angular momentum, \eqref{linear_mom} and \eqref{angular_mom} respectively, equal to it's applied forces and torques (Newton's second law of motion). The forward step starts with link $1$ and continues until link $n$. The imaginary joint zero has no movement so $\omega_0, \alpha_0, a_{c,0}, a_{e,0}$ are all zero. Then solve $\omega_i, \alpha_i, a_{c,i}, a_{e,i}$ from \Crefrange{omega}{a_center} for all links until $n$. This establish the velocity and acceleration of all links with respect to the \glsf{inertialframe}.

\begin{equation}\label{linear_mom}
\dot{p}_i = m_i a_{ci}
\end{equation}

\begin{equation}\label{angular_mom}
\dot{h}_i = I_i \alpha_i + \omega_i \times \left ( I_i \omega_i \right )
\end{equation}

\begin{align}\label{omega}
\begin{split}
\omega_i &= \left ( R^{i-1}_i \right )^T\omega_{i-1} + b_i\dot{q}_i\\
b_i &= \left ( R^0_i \right )^T z_{i-1}
\end{split}
\end{align}

\begin{equation}\label{alpha}
\alpha_i = \left ( R^{i-1}_i \right )^T\alpha_{i-1} + b_i\ddot{q}_i+\omega_i \times b_i\dot{q}_i
\end{equation}

\begin{equation}\label{a_end}
a_{e,i} = \left ( R^{i-1}_i \right )^T a_{e,i-1} + \dot{\omega}_i \times r_{i,i+1}+\omega_i \times \left ( \omega_i \times r_{i,i+1} \right )
\end{equation}

\begin{equation}\label{a_center}
a_{c,i} = \left ( R^{i-1}_i \right )^T a_{e,i-1} + \dot{\omega}_i \times r_{i,ci}+\omega_i \times \left ( \omega_i \times r_{i,ci} \right )
\end{equation}

Lastly, the backward recursion starts with the last link $n$ and works itself backward until link 1. This step finds the force and torque in each link by solving the equilibrium in \Crefrange{linear_equi}{angular_equi}. The change in linear and angular momentum, $\dot{p}_i$ and $\dot{h}_i$, is taken from \eqref{linear_mom} and (\ref{angular_mom}). An overview of the \gls{nef} as an algorithm is presented in \algref{nefAlgo}.

\begin{equation}\label{linear_equi}
\dot{p}_i = f_i - R^i_{i+1} f_{i+1} + m_i g_i
\end{equation}

\begin{equation}\label{angular_equi}
\dot{h}_i = \tau_i - R^i_{i+1}\tau_{i+1} + f_i \times r_{i,ci}-\left ( R^i_{i+1}f_{i+1} \right ) \times r_{i+1,ci}
\end{equation}

An opposing alternative to the Newton-Euler method is the Euler-Lagrange formulation. This method will consider the manipulator as a whole while the Newton-Euler method has a distinct link by link recursive procedure. Because of the recursion in Newton-Euler, it is easier to implement and is known to be generally faster \cite{Hollerbach1980}.

\clearpage

\begin{algorithm}[H]
\DontPrintSemicolon
	\textbf{Forward step}\;
	$\omega_0, \alpha_0, a_{c,0}, a_{e,0} = 0$  \tcp*[f]{joint 0 is always fixed}\;
	 \For{link i=1,2,...,n}{
		  solve $\omega_i$ from (\ref{omega})\;
		  solve $\alpha$ from (\ref{alpha})\;
		  solve $a_{e,i}$ from (\ref{a_end})\;
		  solve $a_{c,i}$ from (\ref{a_center})\;
	 }
	 \BlankLine
	\textbf{Backward step}\;
	$f_{n+1}, \tau_{n+1} = 0$  \tcp*[f]{if no load at end effector}\;
	$f_{n+1}, \tau_{n+1} = f_e, \tau_e$  \tcp*[f]{if load at end effector}\;
	 \For{link i=n,n-1,...,1}{
		  solve $f_i$ from (\ref{linear_equi})\;
		  solve $\tau_i$ from (\ref{angular_equi})\;
	 }
	 
 \caption{Newton-Euler method}\label{nefAlgo}
\end{algorithm}





