\chapter{Previous work}\label{previous_work}

\todo[inline]{
\textbf{Todo}
- Move GeoMod section to this position?
}

\section{T-group}


\todo[inline]{
\textbf{Todo}

- Where is mechanics calculated as for the robot with forces in joints?
}

The \texttt{Tgroup} class is a central part of the GeoMod program. A simplified version of the class can be seen in \liref{Tgroup}. Tgroup stands for transformation group and holds a collection central parameters for a part like center of gravity (line 3) and frame (line 4). The latter is called extended basis in GeoMod. It also holds a list of coordinates for the geometry (line 7).

\lstinputlisting[caption={\texttt{tgroup.h}, simplified version}]{./snippet/Tgroup.h}\label{Tgroup}

%\lstinputlisting[caption={\texttt{extbas.h}, simplified version}]{./snippet/ExtBasis.h}\label{ExtBasis}

\section{Kinematics}

\todo[inline]{
\textbf{Todo}

- How is forward and inverse kinematics implemented?

- How does the classes Modeldata, Machine and arm1 / robot work together?
}

A chained robot is made up of multiple parts described by the class \texttt{Part}, \liref{Part}. These parts could be the different parts of the chained robot in \figref{geomod_interface}. Each of the parts can be rotated with direct kinematics (line 9).

\todo[inline]{
\textbf{Todo}

- Describe/ show overview of how kinematics work in GeoMod
}

\lstinputlisting[caption={\texttt{part.h}, simplified version}]{./snippet/Part.h}\label{Part}


\section{Dynamics}
\todo[inline]{
\textbf{Todo}

- Show robot with moment and joint forces example

- Is time stored in any way in the current version?

- What is future\_basis and past\_basis in Tgroup?

- What are the limitations of the current implementation?

}