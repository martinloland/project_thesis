\chapter{Previous work}\label{previous_work}

\section{T-group}

\todo[inline]{
\textbf{Todo}

- Explain the main workings of the T-group

- How does it link computer graphics and mechanics
}

\section{Kinematics}

\todo[inline]{
\textbf{Todo}

- With reference to the T-group, how:

- How is rotation, position and translation defined and controlled in GeoMod now?

- Show the class structure that holds this information

- How is forward and inverse kinematics implemented?

- How is the ability to have fluid motion?
}

\section{Visualization}

\todo[inline]{
\textbf{Todo}

- With reference to the T-group, how:

- What is the main procedure for going from model geometry to view on screen?

- In what way are the visualization and the physical calculations linked?

- How are rotation and translation of geometry handled?
}

\section{Dynamics}
\todo[inline]{
\textbf{Todo}

- Today's ability for modelling physical properties?

- Robot with moment and joint forces example

- What are the limitations of the current implementation?

- What is the next step or do we have to take a few steps backwards and create a new framework to create support for future functionality?

- Show the parts of the class structure that handles this information

- the attempts of controlling a real robotic arm has provided mixed results, with a more robust framework for modelling dynamic behaviour and join limitations the system might be able to take these factors into account
}