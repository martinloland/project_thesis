\makeglossaries

\newglossaryentry{frame}
{
    name={frame},
    description={a three-dimensional coordinate system which is defined with a position and orientation}
}

\newglossaryentry{inertialframe}
{
    name={inertial reference frame},
    description={reference frame which does not experience any acceleration}
}
 
 
\newglossaryentry{bodyframe}
{
    name={body attached frame},
    description={reference frame for link i with it's origin in the mass center and orientation parallel to the link}
}

\newglossaryentry{DH}
{
    name={Denavit-Hartenberg convention},
    description={convention for describing a link in a serial manipulator robot}
}

\newglossaryentry{homo}
{
    name={homogeneous transformation},
    description={transformation that describes a combination of purely rotation and translation}
}


\newglossaryentry{nef}
{
    name={Newton-Euler method},
    description={A forward-backward recursion method to solve for forces and torques in a linked body system given a position, velocity and acceleration}
}

\newglossaryentry{endeff}
{
    name={end effector},
    description={last part of a chained robot arm, gripper or similar used for interaction with objects}
}

\newglossaryentry{6R}
{
    name={6R robot},
    description={an industrial manipulator with six revolute joints}
}

\newglossaryentry{workspace}
{
    name={workspace},
    description={the working volume of the robot, points where the robot can not reach are outside the workspace of the robot}
}

\newglossaryentry{invkin}
{
    name={inverse kinematic},
    description={calculations that tries to find the joint angles given a wanted position and orientation of the end effector}
}


\newglossaryentry{dirkin}
{
    name={direct kinematic},
    description={calculations that finds the position and orientation of the end effector given a series of joint angles}
}

